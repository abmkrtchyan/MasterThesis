{
    \clearpage
    \section{Ծրագրային կոդի հատկությունների հարցումների համակարգի նախագծում}\label{sec:queryEngineDesign}
    Ծրագրային կոդի հատկությունների հարցումների համակարգի նախագծումը բաղկացած է երկու հիմնական փուլերից (Նկար \ref{fig:figure2})`
    \begin{enumerate}
        \item տվյալների հավաքագրում,
        \item հարցումների համակարգ։
    \end{enumerate}

    \begin{figure}[h]
        \centering
        \includegraphics[width=1\textwidth]{pic2}
        \caption{Հարցումների համակարգի փուլերը}
        \label{fig:figure2}
    \end{figure}

    Առաջին փուլը տվյալների հավաքագրման փուլն է, որի նպատակն է ստեղծել տվյալների բազա, որը կպահի աղբյուր կոդի
    վերաբերյալ անհրաժեշտ տեղեկությունները:
    Բազմաթիվ բաց կոդով նախագծերից իրականացվում է տվյալների հավաքագրում.
    աղբյուր կոդից ստանալով անհրաժեշտ տեղեկություններ պահպանում ենք դրանք տվյալների բազայում:

    Երկրորդ փուլը բուն hարցումների համակարգի նախագծումն է: Այս փուլում մշակվում է հարցումների համակարգ, որի միջոցով
    օգտագործողները կատարելով հարցումներ կարող են վերլուծել իրենց ծրագրերի հատկությունները:

    {
    \subsection{Տվյալների հավաքագրում}\label{subsec:dataCollection}
    Ծրագրային կոդի հատկությունների հարցումների համակարգի համար էական դեր ունի տվյալների հավաքագրման փուլը
    (Նկար \ref{fig:figure5}): Այս փուլում  կարևոր է տվյալներ հավաքագրող համակարգի
    օգտագործումը: Տվյալներ հավաքագրող համակարգի նպատակն է վերլուծել ծրագրային կոդը և հավաքագրել անհրաժեշտ
    տեղեկություններ՝ ներառյալ ծրագրի կառուցվածքի, ղեկավարման հոսքի և ներքին փոխկապակցվածությունների վերաբերյալ։

    \begin{figure}[h]
        \centering
        \includegraphics[width=1\textwidth]{pic5}
        \caption{Փուլ 1, տվյալների հավաքագրում}
        \label{fig:figure5}
    \end{figure}

    Տվյալներ հավաքագրող համակարգը մուտքում ստանում է աղբյուր կոդի ֆայլերը և իրականացնում վերլուծություններ՝
    ընդգրկելով հետևյալ տեղեկությունների հավաքագրումը.
    \begin{enumerate}
        \item Դասերի մասին տեղեկություններ՝ ներառյալ յուրաքանչյուր դասի անունը, դասի դաշտերը,
        աղբյուր կոդի տողերի համարները, ժառանգականության կապերը, ֆունկցիաները և այլն,
        \item Ֆունկցիաների մասին տեղեկություններ՝ ներառյալ յուրաքանչյուր ֆունկցիայի անունը, վերասահմանված ֆունկցիա լինելը,
        աղբյուր կոդի տողերի համարները, հայտարարող դասի անունը, կանչվող ֆունկցիաները, արգումենտները, լոկալ փոփոխականները,
        օգտագործվող և սահմանվող դաշտերը, տեսանելիությունը և այլն,
        \item Հրահանգների մասին տեղեկություններ՝ ներառյալ յուրաքանչյուր հրահանգների տիպը, աղբյուր կոդի տողերը,
        կանչվող ֆունկցիաները, օպերանդները, օգտագործվող և սահմանվող փոփոխականները և այլն:
    \end{enumerate}

    Այս տվյալները պահվում են կառուցված տվյալների բազայում՝ հետագա վերլուծությունների և հարցումների համար
    օգտագործելու նպատակով: Դրանք հիմք են հանդիսանում ծրագրային կոդի հատկությունների հարցումների համակարգի համար:

    {
    \subsubsection{Տվյալների բազայի նախագծում}\label{subsubsec:database}

    Հարցումները արդյունավետ կատարելու նպատակով նախագծվել է տվյալների բազա՝ հաշվի առնելով ռելացիոն ու ոչ ռելացիոն
    բազաների հատկությունները։ Տվյալների բազան ներառում է աղբյուր կոդի հիմնական տարրերի՝ դասերի, դասերի անդամ դաշտերի
    և հրահանգների մասին տեղեկություններ (Նկար \ref{fig:figure7}):

    \begin{figure}[h]
        \centering
        \includegraphics[width=0.8\textwidth]{pic7}
        \caption{Տվյալների բազայի կառուցվածքը}
        \label{fig:figure7}
    \end{figure}
}

%    {
    \subsubsection{Բաց կոդով հասանելի նախագծերի հավաքագրում}\label{subsubsec:sourceCodes}
}
}

    {
    \subsection{Հարցումների համակարգ}\label{subsec:queries}

    Երկրորդ փուլում նախագծվել է բուն հարցումների համակարգը:
    Նախագծված համակարգը հանդիսանում է ծրագրային կոդի հատկությունների վերլուծության կարևոր բաղկացուցիչ մասը:
    Հարցումների համակարգը ճկուն է, ընդլայնելի և հեշտ օգտագործվող:
    Այս համակարգը թույլ է տալիս ծրագրավորողներին, նույնիսկ առանց խորը վերլուծական գիտելիքների,
    արագ և ճշգրիտ ստանալ տեղեկատվություն իրենց ծրագրերի կառուցվածքի և հատկությունների մասին:

    Նախագծված API-ի միջոցով օգտագործողները կարող են կատարել հարցումներ, որի արդյունքում համակարգը տվյալների բազայից վերցնում է անհրաժեշտ
    տվյալներ, կատարում համապատասխան վերլուծություններ և տալիս է հարցմումների պատասխանները (Նկար \ref{fig:figure6}):

    \begin{figure}[h]
        \centering
        \includegraphics[width=1\textwidth]{pic6}
        \caption{Փուլ 2, Հարցումների համակարգ}
        \label{fig:figure6}
    \end{figure}

    {
    Գրել բաժանումների մասին
    \subsubsection*{Կլասներ}
    Կլասների հարցումների մասին խոսել։

    \subsubsection*{Ֆունկցիաներ}
    Ֆունկցիաների հարցումների մասին խոսել։

    \subsubsection*{Ինստրուկցիաներ}
    Ինստրուկցիաների հարցումների մասին խոսել։

    \subsubsection*{Վիճակի փոփոխության վերլուծություն}
    Ավելացնել ալգորիթմերի բարդությունը։
}
}
}
