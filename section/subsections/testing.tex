{
    \subsection{Թեստավորում}\label{subsec:testing}
    Մշակված գործիքը թեստավորվել և համեմատվել է աշխարհում արդեն գոյություն ունեցող այլ գործիքների հետ։ Ինչպես նաև գործիքի
    միջոցով թեստավորվել են բաց կոդով հասանելի ավելի քան 100 պրոեկտներ, որոնք իրականացված են մեծամասամբ C ծրագրավորման լեզվով։

    \subsubsection{Արդյունքների համեմատումը Juliet թեստերի հավաքածույի վրա}
    MLH-ը թեստավորվել է Juliet թեստերի հավաքածույի վրա, որը նախատեսված է ծրագրային ապահովման գործիքների թեստավորման և
    արդյունքների գնահատման համար։ Այն Software Assurance Metrics and Tool Evaluation(SAMATE)\cite{SAMATE} նախագծի մասն է
    կազմում, որը մշակվել է National Institue of Standards and Technology(NIST)-ի կողմից։ Ջուլիետ թեստերի հավաքածուն
    ներառում է ավելի քան 120,000 թեստերի օրինակներ առանձնացված զանազան խնդիրների համար ինչպիսիք են դինամիկ հիշողության
    արտահոսքը, բուֆերի գերհագեցումը և այլն։ Այն ներառում է թեստեր C/C++, Java և Ada լեզուներով:
    Թեստավորման համար հավաքածուից առանձնացվել են միայն դինամիկ հիշողության արտահոսքի համար նախատեսված
    օրինակները (CWE401\_Memory\_Leak\cite{CWE401}): Մշակվել է թեստավորման համակարգ, որը գնահատում է ունեցած գործիքների
    արդյունավետությունը վերը նշված հավաքածուի համար։
}