{
	A procedure graph is a type of control flow graph that represents the flow of control within a single procedure or function in a program. It is a directed graph where each node represents a basic block, which is a straight-line piece of code without any jumps or branches. The edges between nodes represent possible control flow paths between basic blocks. 
	
	Procedure graphs are used in static analysis to analyze the behavior of individual procedures or functions in a program. They can be used to identify potential coding errors, security vulnerabilities, or performance issues. By analyzing the procedure graph, it is possible to determine the possible paths that control flow can take through the code, and identify potential problematic or inefficient patterns of control flow. 
	
	In your research, you mentioned that you have developed a static analysis tool that creates procedure graphs using source code, and uses a taint engine to analyze data flow dependencies between different graphs. This allows you to identify potential security vulnerabilities in the code by tracking the flow of tainted data from sources to sinks, and detecting any problematic data flow dependencies that could indicate a security risk.
	
	\subsubsection{Տվյալների հոսքի վերլուծություն}
	
	Տվյալների հոսքի վերլուծությունը ծրագրի կատարման ճանապարհներին տվյալների հոսքերի մասին ինֆորմացիայի հավաքագրումն է։ Այն կատարվում է ղեկավարման հոսքի գրաֆի հիման վրա։ Տվյալների հոսքի վերլուծության երկու հիմնական մեթոդներն են ակտիվ փոփոխակաների և հասնող սահմանումների\cite{aho} վերլուծությունները։ Գործիքում տվյալների հոսքի վերլուծությունը կատարվել է  ակտիվ փոփոխականների վերլուծությամբ։
	
	\subsubsection{Ակտիվ փոփոխականների վերլուծություն}
	
	Փոփոխականն ակտիվ է ծրագրի որոշակի կետում, եթե իրեն վերագրված արժեքը ղեկավարման հոսքի գրաֆում իր ժառանգների կողմից օգտագործվում է:
	Վերլուծություն կատարելու համար օգտագործում ենք հետևյալ չորս բազմությունները՝ \textit{def}, \textit{use}, \textit{in} և \textit{out}։ Ղեկավարման հոսքի V գագաթի def բազմությունը պարունակում է այն փոփոխականները, որոնք արժեքավորվել են V գագաթում, իսկ use բազմությունը՝ որոնք օգտագործվել են V գագաթում։ Ունենալով այս երկու բազմությունները, կարող ենք հաշվել in և out բազմությունները (այսինքն ակտիվ փոփոխականների բազմությունը գագաթ մուտք գործելիս և գագաթից դուրս գալիս), հետևյալ հավասարումների միջոցով\cite{aho}.
	
	{		
		\vspace{-\baselineskip}	
		\begin{align*}
			&\text{IN}[exit] = \varnothing & \\
			&\text{IN}[V] = \text{use}[V] \cup (\text{OUT}[V] \setminus \text{def}[V]) & \\
			&\text{OUT}[V] = \bigcup_{s \in \text{V}{successors}} \text{IN}[s] & 
		\end{align*}
	}

	Նկատենք, որ in և out բազմությունները հաշվելու հավասարումները ռեկուրսիվ են և փոխկապակցված։ Առաջին հավասարումը սահմանում է սահմանային պայմանը։ Սա նշանակում է, որ ծրագրի ավարտին ակտիվ փոփոխականներ չկան։ Երկրորդ հավասարումով ասվում է, որ փոփոխականը գագաթ մուտք գործելիս ակտիվ է, եթե այն օգտագործվում է գագաթում, կամ դուրս է գալիս գագաթից առանց վերասահմանվելու։ Երրորդ հավասարումով ասվում է, որ գագաթից դուրս գալիս փոփոխականն ակտիվ է այն և միայն այն դեպքում, երբ այն ակտիվ է իր ժառանգ գագաթներից գոնե մեկում։
	
	\subsubsection{def և use բազմություններ}
	
	Յուրաքանչյուր գագաթում առկա փոփոխականները դիտարկվում են կամ  որպես սահմանվող(def) կամ որպես օգտագործվող(use) փոփոխականներ։ Որոշ  փոփոխականներ դիտարկվում են և որպես սահմանվող, և որպես օգտագործվող փոփոխակներ։ Օրինակ, եթե դիտարկենք “++” post increment օպերատորը, “var++” արտահայտությունում var փոփոխականը և օգտագործվում է և նորից սահամանվում։ Դիտարկված փոփոխականներն ավելացվում են def և use բազմություններում։ \\
	Այսպիսով կատարվում է def և use փոփոխականների վերլուծություն և ղեկավարման հոսքի գրաֆի յուրաքանչյուր գագաթի համար տրվում է այդ գագաթի def և use բազմությունները։
	
}