{
	\thispagestyle{plain}
	\begin{center}
	    \large
	    \textbf{Համառոտագիր}
	\end{center}
	
	\begin{flushleft}
		\vspace{0.5cm}
		
		\large
		\textbf{Ծրագրային կոդի հատկությունների հարցումների համակարգ}
		
		\vspace{0.2cm}
		\textbf{Система для запросов свойств исходного кода}
		
		\vspace{0.2cm}
		\textbf{A system for querying properties of source code}
		
		\vspace{0.5cm}
		
	\end{flushleft}
	Անընդհատ աճող ծրագրային ապահովման ծավալների հետ մեկտեղ առաջանում է ծրագրերի ավտոմատ վերլուծության գործիքների պահանջարկ։ Այդպիսի գործիքների շարքին են պատկանում նախնական ու կատարվող կոդի համապատասխանեցման գործիքները։ Այս գործիքները օգտագործվում են հեղինակային իրավունքների խախտումների, ծրագրային խոցելիությունների հայտնաբերման, reverse engineering-ում առաջացող խնդիրների լուծման համար։ Աշխատանքի շրջանակներում հետազոտվել են ժամանակակից նախնական և կատարվող կոդի համապատասխանությունը գտնող գործիքները։ Գոյություն ունեցող գործիքները չեն տալիս նախնական ու կատարվող կոդի հրամանների համապատասխանեցում։ Հետազոտված գործիքները հաշվի են առնում ծրագրի հրամանների ու հատկությունների մի մասը։ Աշխատանքում նախագծվել ու իրականացվել է գործիք, որը թույլ է տալիս համապատասխանեցնել նախնական ու կատարվող կոդի հրամանները։ Ինչպես նաև գոյություն ունեցող գործիքներից մեկում իրականացվել են փոփոխություններ, որի ճշտությունը։
}