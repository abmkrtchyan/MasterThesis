{
	\thispagestyle{plain}
	\begin{center}
		\large
		\textbf{Համառոտագիր}
	\end{center}

	\begin{center}
		\vspace{0.5cm}

		\large
		\textbf{Ծրագրային կոդի հատկությունների հարցումների համակարգ հիմնված ղեկավարման կախվածությունների գրաֆի վրա}

		\vspace{0.2cm}
		\textbf{Система запроса свойств исходного кода на основе графа зависимостей управления}

		\vspace{0.2cm}
		\textbf{A system for querying source code properties based on a control dependency graph}

		\vspace{0.5cm}

	\end{center}
	Աշխատանքի շրջանակներում հետազոտվել են ծրագրային կոդի վերլուծություն կատարող ժամանակակից գործիքները։ Գոյություն ունեցող
	գործիքների մեծ մասը կարողանում են հայտնաբերել միայն սահմանափակ քանակի խնդիրներ, և չեն ապահովում բավարար
	ֆունկցիոնալություն: Ֆունկցիոնալության ավելացման համար անհրաժեշտ են խորը գիտելիքներ ծրագրային վերլուծության
	բնագավառում, սակայն շատ ծրագրավորողներ չեն տիրապետում նմանատիպ վերլուծությունների։
	Աշխատանքում նախագծվել և իրականացվել է նոր գործիք, որը պահանջելով միայն համապատասխան ծրագրավորման լեզվի իմացություն
	թույլ է տալիս կատարել ծրագրային կոդի վերլուծություն։
}