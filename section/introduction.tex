{
    Ծրագրային ապահովման ոլորտը շարունակաբար զարգանում և ընդլայնվում է, ինչի արդյունքում ծրագրային ապահովման ծավալներն
    անընդհատ աճում են: Անընդհատ աճող ծրագրային ապահովման ծավալների հետ մեկտեղ առաջանում է ծրագրային կոդի հատկությունների
    ավտոմատ վերլուծության գործիքների պահանջարկ: Ավտոմատացված վերլուծությունները թույլ են տալիս բացահայտել ծրագրի տարբեր
    կողմերը, ինչպիսիք են դասերի, ֆունկցիաների, փոփոխականների և հրահանգների կապերը, ինչպես նաև ստանալ տեղեկատվություն
    դրանց մասին: Նման վերլուծությունները կարևոր են ոչ միայն նոր ծրագրերի մշակման, այլև գոյություն ունեցող ծրագրերի
    պահպանման և զարգացման համար:
}