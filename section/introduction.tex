{
    Ծրագրային ապահովման ոլորտը շարունակաբար զարգանում և ընդլայնվում է, ինչի արդյունքում ծրագրային ապահովման ծավալներն անընդհատ աճում են: Անընդհատ աճող ծրագրային ապահովման ծավալների հետ մեկտեղ առաջանում է ծրագրային կոդի հատկությունների ավտոմատ վերլուծության գործիքների պահանջարկ: Ավտոմատացված վերլուծությունները թույլ են տալիս բացահայտել ծրագրի տարբեր կողմերը, ինչպիսիք են դասերի, ֆունկցիաների, փոփոխականների և հրահանգների կապերը, ինչպես նաև ստանալ տեղեկատվություն դրանց մասին: Նման վերլուծությունները կարևոր են ոչ միայն նոր ծրագրերի մշակման, այլև գոյություն ունեցող ծրագրերի պահպանման և զարգացման համար:

    Աշխարհում գոյություն ունեցող ծրագրային կոդի հատկությունների վերլուծության գործիքները, սակայն, ունեն որոշ սահմանափակումներ և թերություններ: Այդ գործիքների մեծ մասը կարողանում են հայտնաբերել միայն սահմանափակ քանակի խնդիրներ և չեն ապահովում բավարար ֆունկցիոնալություն: Ֆունկցիոնալության ավելացման համար անհրաժեշտ են խորը գիտելիքներ ծրագրային վերլուծության բնագավառում, սակայն շատ ծրագրավորողներ չեն տիրապետում նմանատիպ վերլուծությունների, իսկ այդ գիտելիքները ձեռք բերելու համար անհրաժեշտ են ժամանակ և ռեսուրսներ:

    Ուստի, առաջանում է ծրագրային կոդի հատկությունների վերլուծության այնպիսի համակարգի անհրաժեշտություն, որը կարող է մատչելի լինել ծրագրավորողներին` առանց խորը մասնագիտական գիտելիքների պահանջի: Այս մագիստրոսական թեզի նպատակն է նախագծել և իրականացնել ծրագրային կոդի հատկությունների հարցումների համակարգ, որը կարող է համեմատվել GitHub CodeQL-ի հետ և կօգնի ծրագրավորողներին ավելի արդյունավետ հասկանալ և պահպանել իրենց ծրագրերը: Թեզի շրջանակում կմշակվի նաև դինամիկ հիշողության արտահոսքի սխալների հայտնաբերման մեխանիզմ, որը կօգնի բացահայտել և վերացնել այդպիսի սխալները:

    Ծրագրային ապահովման ոլորտում ծրագրային կոդի հատկությունների վերլուծության և դրա մասին տեղեկատվության տրամադրման կարևորությունը շարունակաբար աճում է: Այս գործընթացը թույլ է տալիս բացահայտել ծրագրի տարբեր կողմերը, ինչպիսիք են դասերի, ֆունկցիաների, փոփոխականների և հրահանգների կապերը, ինչպես նաև ստանալ տեղեկատվություն դրանց մասին: Նման վերլուծությունները կարևոր են ոչ միայն նոր ծրագրերի մշակման, այլև գոյություն ունեցող ծրագրերի պահպանման և զարգացման համար:

    Աշխարհում առկա ծրագրային կոդի հատկությունների վերլուծության գործիքներն, այնուամենայնիվ, ունեն որոշ սահմանափակումներ և թերություններ: Այդ գործիքների մեծ մասը կարողանում են հայտնաբերել միայն սահմանափակ քանակի խնդիրներ և չեն ապահովում բավարար ֆունկցիոնալություն: Ֆունկցիոնալության ավելացման համար անհրաժեշտ են խորը գիտելիքներ ծրագրային վերլուծության բնագավառում, սակայն շատ ծրագրավորողներ չեն տիրապետում նմանատիպ վերլուծությունների, իսկ այդ գիտելիքները ձեռք բերելու համար անհրաժեշտ են ժամանակ և ռեսուրսներ:

    Այս խնդիրը լուծելու համար առաջանում է ծրագրային կոդի հատկությունների վերլուծության այնպիսի համակարգի անհրաժեշտություն, որը կարող է մատչելի լինել ծրագրավորողներին` առանց խորը մասնագիտական գիտելիքների պահանջի: Այս մագիստրոսական թեզի նպատակն է նախագծել և իրականացնել ծրագրային կոդի հատկությունների հարցումների համակարգ, որը կարող է համեմատվել GitHub CodeQL-ի հետ և կօգնի ծրագրավորողներին ավելի արդյունավետ հասկանալ և պահպանել իրենց ծրագրերը: Թեզի շրջանակում կմշակվի նաև դինամիկ հիշողության արտահոսքի սխալների հայտնաբերման մեխանիզմ, որը կօգնի բացահայտել և վերացնել այդպիսի սխալները:

    Ծրագրավորողների համար կարևոր է ունենալ գործիքներ, որոնք կօգնեն ավելի լավ հասկանալ և վերլուծել իրենց ծրագրերի կառուցվածքը, ֆունկցիոնալությունը և հատկությունները: Ծրագրային կոդի հատկությունների հարցումների համակարգերը հզոր գործիքներ են, որոնք կարող են մեծ օգնություն ցուցաբերել ծրագրավորողներին իրենց աշխատանքում:

    Այդպիսի համակարգերը թույլ են տալիս ավտոմատացնել ծրագրի վերլուծությունը, ստանալ տեղեկատվություն դասերի, ֆունկցիաների, փոփոխականների և դրանց փոխկապակցվածության մասին, ինչպես նաև հայտնաբերել հիշողության արտահոսքի սխալները: Սակայն առկա ծրագրային կոդի հատկությունների վերլուծության գործիքները չեն կարողանում բացահայտել ծրագրի կառուցվածքի և գործունեության վերաբերյալ ամբողջական պատկեր: Այդ գործիքներն ունեն սահմանափակ հայտնաբերման ներուժ և սահմանափակ ֆունկցիոնալություն:

    Այս սահմանափակումները դժվարացնում են ծրագրավորողների աշխատանքը, քանի որ նրանք չեն կարողանում ստանալ ամբողջական և խորը վերլուծություններ իրենց ծրագրերի մասին: Այս պատճառով էլ առաջանում է ավելի հզոր և համապարփակ ծրագրային կոդի վերլուծության համակարգի անհրաժեշտություն:

    Այս մագիստրոսական թեզի նպատակն է նախագծել և իրականացնել ծրագրային կոդի հատկությունների հարցումների համակարգ, որը կարող է համեմատվել GitHub CodeQL-ի հետ և կօգնի ծրագրավորողներին ավելի արդյունավետ հասկանալ և պահպանել իրենց ծրագրերը: Թեզի շրջանակում կմշակվի նաև դինամիկ հիշողության արտահոսքի սխալների հայտնաբերման մեխանիզմ, որը կօգնի բացահայտել և վերացնել այդպիսի սխալները:

    Այս նոր համակարգը կտրամադրի ավելի խորը և համապարփակ վերլուծություններ ծրագրի կառուցվածքի, գործունեության և հատկությունների վերաբերյալ: Դա կօգնի ծրագրավորողներին ավելի լավ հասկանալ և պահպանել իրենց ծրագրերը` առանց խորը մասնագիտական գիտելիքների պահանջի:
}