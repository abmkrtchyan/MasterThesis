\subsection{Sparrow}
Sparrow\cite{Jung2008}-ն ստատիկ վերլուծություն կատարող գործիք է, որը հայտնաբերում է դինամիկ հիշողության արտահոսքի խնդիրը
С լեզվով գրված ծրագրերում։ Այն հիմնված է վերացական մեկնաբանության վրա և կարգավորում է ցիկլերը, կամայական ֆունկցիայի
կանչերի ցիկլերը և կեղծանունները: Ալգորիթմը հուսալի չէ։ Այսպիսով, այն չի կարող երաշխավորել ծրագրում հնարավոր բոլոր
հիշողության արտահոսքների հայտնաբերումը:

Sparrow-ն առանձին վերլուծում է յուրաքանչյուր ֆունկցիայի հիշողության օգտագործման վարքը և ստացված ինֆորմացիան պահպանում:
Յուրաքանչյուր ֆունկցիայի նկարագիրն օգտագործվում է իր կանչերի ժամանակ: Այն վերլուծում է ֆունկցիաների կանչերի գրաֆը հակառակ
տեղաբանական կարգով: Ֆունկցիայի կանչերի ցիկլերի դեպքում կանչերի ցիկլի բոլոր ֆունկցիաները միասին վերլուծվում են մեկ ֆիքսված
կետի կրկնության շրջանակներում: Դինամիկ կանչի սահմանների դեպքում (օրինակ ՝ ֆունկցիայի ցուցիչների պատճառով) ֆունկցիան
կանչողը սպասում է, մինչև կանչվող ֆունկցիայի վերլուծությունը պատրաստ լինի:

Ֆունկցիաների վերլուծության ամփոփումը կազմված է երկու հիմնական փուլերից։ (1) Հիշողության էֆեկտների գնահատում և (2)
գնահատման արդյունքների օգտագործում՝ հիշողության հնարավոր արտահոսքերը հայտնաբերելու համար օգտակար տեղեկատվությունից
բաղկացած ամփոփում ստանալու նպատակով: Հիշողության էֆեկտի գնահատման փուլը հիմնված է վերացական մեկնաբանության վրա։

Յուրաքանչյուր ֆունկցիայի համար հիշողության էֆեկտը բաղկացած է տեղեկատվության երեք մասից՝ առանձնացված հասցեներ, ազատված
հասցեներ և ֆունկցիայի ավարտին հիշողության կարգավիճակը։ Գործիքը հավաքում է բոլոր այն հասցեները, որոնք ակնհայտ
հասանելիություն ունեն դրսից (օրինակ գլոբալ փոփոխականներ, ցուցիչ արգումենտներ, վերադարձվող արժեք և այլն)։ Այնուհետև
գործիքը գտնում է, թե դրանցից որ հասցեներն են առանձացված հիշողության հասցեներ, որոնք ազատված կամ վերանվանված։ Արդյունքները
պահպանվում են տվյալ ֆունկցիայի ամփոփման մեջ։

Յուրաքանչյուր ֆունկցիայի վերլուծության ամփոփում իր մեջ ներառում է տեղեկություն դրսից հասանլեի բոլոր հասցեների մասին։
Գործիքը տեղեկացնում է դինամիկ հիշողության արտահոսքի սխալի հայտնաբերման մասին, երբ ֆունկցիայի վերլուծության արդյունքում
գտնվում է դինամիկ հիշողության օբյեկտ, որի հասցեավորումը կորցնում է տվյալ ֆունկցիան և դրսից ոչ մի տեղ չի պահպանվում։
Ստորև ներկայացված են գործիքի որոշ թերություններ՝
\begin{enumerate}
    \item Բոլոր գլոբալ փոփոխականների մասին ինֆորմացիան պահվում է գրաֆի մեկ հանգույցում և այդպիսով գործիքը չի բացահայտում
    միևնույն գլոբալ փոփոխականի մեջ տարբեր հիշողության օբյեկտների հասցեներ պահպանելու խնդիրը։
    \item Վերլուծությունն անզգայուն է ճանապարհների նկատմամբ: Ֆունկցիայի արգումենտների համար այն հավաքում է բոլոր ազատված
    հասցեները՝ առանց ճանապարհները հաշվի առնելու: Այսպիսով, գործիքը կարող է չհայտնաբերել հիշողության արտահոսքը,
    եթե հիշողության բլոկի արտահոսքը ազատվում է այլ ճանապարհով, քան օգտագործվածը: Նմանապես, գլոբալ փոփոխականների
    նշանակումները հավաքվում են անկախ առանց ճանապարհները հաշվի առնելու:
    \item Գործիքը վերլուծության ընթացքում ենթադրում է, որ բոլոր ցիկլերը ծրագրի աշխատանքի ընթացքում աշխատելու են ամենաքիչը
    մեկ անգամ, ինչը կարող է հանգեցնել կեղծ բացասական արդյունքների։
\end{enumerate}

Փաստաթղթերի համաձայն SPEC2000 հենանիշի և այլ բաց կոդով ծրագրերի վերլուծության արդյունքում գործիքի կողմից հայտնաբերել է
հիշողության 332 արտահոսք, և միայն 47-ը կեղծ դրական են:
