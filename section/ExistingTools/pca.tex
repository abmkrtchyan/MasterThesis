\subsection{PCA}
CA\cite{Li2020}-ն առաջին հերթին թարգմանում է նախնական կոդը LLVM IR-ի։ Այնուհետև, օգտագործելով LLVM-ի gold plugin-ը
բոլոր IR ֆայլերը հավաքում է մեկում։ Ապա այն օգտագործում է Անդերսենի ցուցիչների վերլուծության գործիքը\cite{Andersen}
և հավաքում ցուցիչների մասին ինֆորմացիա, մասնավորապես, թե որ դինամիկ հիշողության օբյեկտի վրա է հղված այս կամ այն ցուցիչը։
Այդ ինֆորմացիայի վրա հիմնվելով, կառուցվում են ֆունկցիաների կանչերի կախվածության և ղեկավարման կախվածության գրաֆները։
Եւ վերջում, արդեն ունենալով համապատասխան գրաֆներն ու ինֆորմացիան, կառուցվում է միջֆունկցիոնալ տվյալների կախվածության գրաֆ (DDG)։

Դինամիկ հիշողության արտահոսքի հայտնաբերման նպատակով PCA-ը յուրաքանչյուր դինամիկ հիշողություն առանձնացնող ինստրուկցիայի (A)
համար հավաքում է բոլոր նրանից հասանելի գագաթները (N) DDG-ում: Եթե N-ը առանձնացված հիշողությունն ազատող ինստրուկցիա
չի ներառում, ապա պնդում է, որ տեղի ունի հիշողության արտահոսք։

Որոշ դինամիկ հիշողության օբյեկտների կարող են հետևել մեկից ավելի դրանք ազատող ինստրուկցիաներ DDG-ում։
Նման դեպքերում, A-ն կհամարվի ազատված, եթե գոյություն ունի ղեկավարման կախվածության ճանապարհ A-ից դեպի այն ազատող
ինստրուկցիան։ Հակառակ դեպքում նույնպես պնդում է, որ տեղի ունի հիշողության արտահոսք։ Գործիքն ունի հետևյալ սահմանափակումները՝
\begin{enumerate}
    \item Վերլուծությունն արվում է հոսքի, դաշտի և համատեքստի հանդեպ ոչ զգայուն։
    \item Այն օգտագործում է LLVM gold plugin-ը IR ֆալերը մեկում հավաքելու համար, այնուհետև կառուցում է DDG, ինչը գործիքը
    դարձնում է ոչ մասշտաբային։
\end{enumerate}

Գործիքի նախնական կոդը հասանելի է\cite{PCA}։
