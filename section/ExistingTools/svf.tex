\subsubsection{SVF}
SVF\cite{Sui2016}-ն հիմնված է LLVM-ի վերին մակարդակում։ Առաջին քայլում այն թարգմանում է նախնական կոդը LLVM IR-ի։
Այնուհետև, օգտագործելով LLVM-ի gold plugin-ը բոլոր IR ֆայլերը հավաքում է մեկում։ SVF-ն օգտագործում է որոշ ցուցիչների
վերլուծիչներ, ցուցիչների համար համապատասխան ինֆորմացիա հավաքելու նպատակով, մասնավորապես, թե որ դինամիկ հիշողության
օբյեկտի վրա է հղված այս կամ այն ցուցիչը։ Վերլուծիչներցի մեկն Անդերսենի ցուցիչների վերլուծիչն է\cite{Andersen}։
Օգտագործելով LLVM IR-ը և ցուցիչների մասին հավաքված ինֆորմացիան՝ գործիքը կառուցում է ղեկավարման կախվածության գրաֆ(CFG)
և հավաքում է հիշողության ստատիկ առանձին վերագրման մասին ինֆորմացիան(SSA - Static Single Assignment)։
Յուրաքանչյուր VFG-ի գագաթ իրենի ներկայացնում է ծրագրի որևէ ինստրուկցիա, իսկ գագաթների միջև կողերը տեղադրվում են
օգտվելով օգտագործման-հայտարարման և ցուցիչների մասին հավաքված ինֆորմացիայից։

Բացի այդ, SVF-ն ապահովում է հիշողության տարածքների տարանջատում, ինչը թույլ է տալիս օգտվողներին հիշողությունը բաժանել
հավաքածուների: Սա օգտակար է մեծածավալ ծրագրերը վերլուծելու համար, եթե հաշվի է առնվում հիշողության որոշակի տարածք:

Տարբեր ստուգման գործիքներ կարող են իրականացվել VFG-ի հիման վրա, որը հասանելի է օգտվող ծրագրերի համար:
Հիշողության արտահոսքի հայտնաբերումը համարվում է աղբյուր-ստացողի խնդիր (յուրաքանչյուր հիշողության հատկացում
յուրաքանչյուր ուղու վրա պետք է հասնի իր ազատմանը): Ստորև նշված են գործիքի որոշ սահմանափակումներ՝
\begin{enumerate}[itemsep=1mm]
    \item Այն օգտագործում է LLVM gold plugin-ը IR ֆալերը մեկում հավաքելու համար, այնուհետև կառուցում է DDG,
    ինչը գործիքը դարձնում է ոչ մասշտաբային։
    \item Վերլուծությունն արվում է ճանապարհների և դաշտերի հանդեպ ոչ զգայուն։
\end{enumerate}