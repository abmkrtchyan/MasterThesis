\subsection{Fastcheck}
Fastcheck\cite{Cherem2007}-ն իրականացնում է միջֆունկցիոնալ վերլուծության ալգորիթմ С լեզվով գրված ծրագրերում դինամիկ
հիշողության արտահոսքի հայտնաբերման համար։ Առաջին քայլով Fastcheck-ը շարահյուսական վերլուծություն է իրականացնում նախնական
կոդի համար ներքին թարգմանիչի միջոցով (այն չի օգտագործում արտաքին այլ թարգմանիչներ) և վերլուծության հիման վրա կառուցում է
ղեկավարման կախվածության գրաֆ (CFG)։ Երկրորդ քայլով այն կատարում է հայտարարման-օգտագործման անալիզ CFG-ի հիման վրա։ Արդեն
հավաքված ինֆորմացիայի հիման վրա այն կատարում է ցուցիչների վերլուծություն (հոսքից անկախ, միավորման վրա հիմնված) և
կառուցում է արժեքների կախվածության գրաֆը (VFG)։ Այս քայլում VFG-ի կողերը դեռ համապատասխանեցված չեն ճյուղավորման
օպերատորների պայմանների հետ։

Ալգորիթմը հետևում է VFG-ի միջոցով հիշողության բաշխման կետերից դեպի ազատման կետեր արժեքների հոսքին: Եթե այդպիսի
ճանապարհներ չկան, ապա հաղորդում է դինամիկ հիշողության արտահոսքի մասին: Հակառակ դեպքում, տեղաբաշխմանը վերաբերող ենթագրաֆը
առանձնացվում է, և այդ ենթագրաֆի կողերը նշվում են համապատասխան ճյուղավորման պայմաններով՝ ստեղծելով պաշտպանված գրաֆիկ:
Հիշողության արտահոսքի հայտնաբերման խնդիրը համապատասխանեցվում է պաշտպանված արժեքների հոսքի գծապատկերում գրաֆի
հասանելիության խնդրին: Եթե կա թույլատրելի ուղի՝ առանց դրանում ազատման ինստրուկցիա կանչելու հրահանգի, ապա հաղորդվում է
հիշողության արտահոսքի մասին: Ճանապարհնեի ստուգումը կատարվում է SMT վերլուծիչի միջոցով: Ստորև նշված են որոշ
սահմանափակումներ՝
\begin{enumerate}
\item Ցուցիչների վերլուծությունը համատեքստի հանդեպ զգայուն չէ։
\item Շարահյուսական վերլուծությունն իրականացվում է ոչ լիարժեք և փորձերի ընթացքում պարզվել է, որ շատ դեպքերում նախնական
կոդը չի հաջողվում վերլուծել։
\end{enumerate}

Fastcheck-ը գրված է Java լեզվով և նախնական կոդը հասանելի է\cite{Fastcheck}:
