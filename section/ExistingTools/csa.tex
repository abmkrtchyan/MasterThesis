\subsubsection{Clang Static Analyzer}
Clang static analyzer\cite{ClangAnalyzer}-ը բաց նախնական կոդով գործիք է, որը հայտնաբերում է С, С++ լեզուներով գրված
ծրագրեում առկա դինամիկ հիշողության արտահոսքի խնդիրները։ Այն իրականացնում է ճանապարհների հանդեպ զգայուն միջֆունկցիոնալ
վերլուծություն։ Գործիքն իրականացնում է ծրագրի սիմվոլիկ կատարում և բոլոր մուտքային տվյալներին տալիս է սիմվոլիկ արժեքներ։

Clang static analyzer-ը Clang\cite{Clang}-ի մի մասն է կազմում և օգտագործում է կառուցվածքներ, վերացական շարահյուսական
ծառեր (AST) և ղեկավարման կախվածության գրաֆ (CFG) մասնատված գրաֆ (EG) կառուցելու նպատակով։ EG-ի գագաթները ծրագրի
կարգավիճակներն են, իսկ կողերը ծրագրի կետերն ու պայմաններն են (ProgramPoint, ProgramState)։ ProgramState-ը նկարագրում է
ծրագրի աշխատանքի վերացական պահը։ (Point1, State1) գագաթից դեպի (Point2, State2) գագաթ կողը պայման է Point1-ի և Point2-ի
միջև, որը State1-ը դարձնում է State2:

Ֆունկցիաների կանչերը հնարավորինս տեղադրված են, քանի որ նրանց համատեքստերն ու ճանապարհները տեսանելի են ֆունկցիան կանչողի
համատեքստում։ Այսպիսով, ուսումնասիրվում են ծրագրի բոլոր հնարավոր ճանապարհները։

Խնդիրները հայտնաբերելու նպատակով, գործիքն օգտագործում է ստուգիչներ, որոնցից յուրաքանչյուրը գեներացնում է համապատասխան
վերացական կարգավիճակներ ծրագրի յուրաքանչյուր կետի համար։ Վերջում դիրարկելով բոլոր կարգավիճակները, ստուգիչը հայտնաբերում
է գոյություն ունեցող խնդիրները։ Գործիքն ունի հետևյալ թերությունները՝
\begin{enumerate}
    \item Ցիկլերի իտերացիաների քանակը հնարավոր է փոխել ‘maxloop’ արգումենտի միջոցով։ Յուրաքանչյուր ցիկլ աշխատում է
    maxloop անգամ և հաստատուն իտերացիաների քանակով ցիկլեր անալիզ անելու ժամանակ կարող է մեծ ինֆորմացիայի կորուստ լինել,
    երբ այդ հաստատունը մեծ լինի maxloop-ից։
    \item Ճանապարհների արագ աճ կարող է տեղի ունենալ մեծաքանակ if-երի դեպքում։ Եւ այդպիսով, էքսպոնենցիալ արագությամբ
    կարող է աճել կարգավիճակների քանակը։
\end{enumerate}

Clang static analyzer-ն LLVM-ի մասն է կազմում և նախնական կոդը հասանելի է\cite{LLVM}։
