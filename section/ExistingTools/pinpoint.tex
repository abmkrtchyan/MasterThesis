\subsubsection{Pinpoint}
Pinpoint\cite{Qingkai2018}-ը փորձում է հաղթահարել արդեն գոյություն ունեցող ցուցիչների վերլուծիչների համար մասշտաբայնության
խնդիրները։
Այն օգտագործվում է մի շարք աղբյուր-ստացող խնդիրների լուծման համար, ինչպիսիք են դինամիկ հիշողության արտահոսքի հայտնաբերման,
ազատումից հետո օգտագործված ցուցիչի հայտնաբերման, կրկնակի ազատման և այլն։ Այս գործիքի հիմնական տրամաբանությունը կայանում է
նրանում, որ ցուցիչների անալիզի փուլը բաժանված է երկու ենթափուլերի։ Առաջինը պարզ ճշգրտությամբ լոկալ փոփոխականների տվյալային
կախվածության հետազոտումն է։ Երկրորդը բարդ և ռեսուրսատար միջֆունկցիոնալ տվյալների կախվածության և ճանապարհների անալիզն է։
Բոլոր փոփոխականների համար վերլուծություն անելու փոխարեն այն ընտրում է միայն անհրաժեշտ փոփոխականները և կատարում է միայն
այդ փոփոխականների միջֆունկցիոնալ վերլուծությունը։ Ճանպարհների իրագործելիության վերլուծությունը կատարվում է ըստ պահանջի
և օգտագործում է SMT վերլուծող համակարգ: Այս մոտեցումը նվազեցնում է կառավարման հոսքի ուղիների քանակը և գործարկում է SMT
վերլուծող համակարգը միայն այդ ուղիների համար: Գործիքն իրականացվել է LLVM-ի վերին մակարդակում և առևտրայնացվել է\cite{Pinpoint}:
