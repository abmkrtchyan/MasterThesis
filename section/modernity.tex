{
    \section{Արդիականություն}\label{sec:modernity}
    Աշխարհում առկա ծրագրային կոդի հատկությունների վերլուծության գործիքներն ունեն որոշ սահմանափակումներ և թերություններ:
    Այդ գործիքների մեծ մասը կարողանում են հայտնաբերել միայն սահմանափակ քանակի խնդիրներ և չեն ապահովում բավարար
    ֆունկցիոնալություն: Ֆունկցիոնալության ավելացման համար անհրաժեշտ են խորը գիտելիքներ ծրագրային վերլուծության
    բնագավառում, սակայն շատ ծրագրավորողներ չեն տիրապետում նմանատիպ վերլուծությունների, իսկ այդ գիտելիքները ձեռք բերելու
    համար անհրաժեշտ են ժամանակ և ռեսուրսներ: Այս խնդիրը լուծելու համար առաջանում է ծրագրային կոդի հատկությունների
    վերլուծության այնպիսի համակարգի անհրաժեշտություն, որը կարող է մատչելի լինել ծրագրավորողներին` առանց խորը մասնագիտական
    գիտելիքների պահանջի:

    Այս աշխատանքի նպատակն է նախագծել և իրականացնել ծրագրային կոդի հատկությունների
    հարցումների համակարգ, որը թույլ կտա կատարել ծրագրային կոդի վերլուծություն՝ պահանջելով միայն համապատասխան ծրագրավորման լեզվի իմացություն։
}
