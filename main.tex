%NOTE! compile with XeLaTeX
\documentclass[a4paper, 12pt]{report}

\usepackage[left=3.0cm, right=1.5cm, top=2.0cm, bottom=2.5cm]{geometry}
\usepackage[utf8]{inputenc}
\usepackage{ragged2e}

\usepackage{lipsum}
\usepackage{enumitem}
\setlist[enumerate]{itemsep=0mm}

\usepackage{graphicx}
\graphicspath{ {./images/} }

\usepackage{titlesec}
\usepackage{tocloft}

\usepackage{setspace}
\usepackage[numbers]{natbib}
\usepackage{fancyhdr}
\usepackage[hidelinks]{hyperref}
\usepackage[none]{hyphenat} %որ տողադարձ չանի

\usepackage{amssymb} % for additional math environments and symbols
\usepackage{amsmath}

%Հայերենի համար անհրաժեշտ բաղադրամասեր
\usepackage{fontspec}
\usepackage{polyglossia}
\setdefaultlanguage[numerals=arabic]{armenian}
\newfontfamily\armenianfont{GHEA Mariam}

% Adjust font size and centering for \bibname
\titleformat{\chapter}[display]
{\bfseries\centering}{\chaptertitlename\ \thesection}{1em}{}

\linespread{1.5} % միջտողային հեռավորությունը
\setlength{\parindent}{0cm} % Նոր պարբերությունը 0սմ հեռավորությունից սկսել

% Բովանդակության համարակալման և տեսքի համար
\renewcommand{\thesection}{\arabic{section}.}
\renewcommand{\thesubsection}{\thesection\arabic{subsection}.}
\renewcommand{\thesubsubsection}{\thesubsection\arabic{subsubsection}.}

% Վերնագրերը տեքստում գրվեն մեջտեղից և մուգ
\titleformat{\section}
{\bfseries\centering}
{\thesection}
{1em}
{}

\titleformat{\subsection}
{\bfseries\centering}
{\thesubsection}
{1em}
{}

\titleformat{\subsubsection}
{\bfseries\centering}
{\thesubsubsection}
{1em}
{}

% Adjust spacing for section headers
\titlespacing*{\section}{0pt}{1cm}{1cm}
\titlespacing*{\subsection}{0pt}{1cm}{1cm}

% Bold fonts for table of contents
\renewcommand{\cftsecfont}{\bfseries}
\renewcommand{\cftsubsecfont}{\bfseries}

\renewcommand{\contentsname}{Բովանդակություն} %որ contents-ի փոխարեն հայերեն գրի
\renewcommand{\cfttoctitlefont}{\bfseries} % որ Բովանդակությունը գրվի փոքր և մուգ տառերով

\renewcommand{\bibname}{Գրականություն} %որ references-ի փոխարեն հայերեն գրի

\sloppy %ձախից և աջից հավասարեցում
\usepackage{indentfirst}
\setlength{\parindent}{1cm} % Գրել մի մատ խորքից

\begin{document}
    \pagenumbering{gobble} % Անջատել էջերի համարակալումը
    \begin{titlepage}
    \begin{center}
        \linespread{1.3}
        \vspace{0.5cm}
        {
            \fontsize{18}{0}
            \textbf{ԵՐԵՎԱՆԻ ՊԵՏԱԿԱՆ ՀԱՄԱԼՍԱՐԱՆ} \\
        }
        \vspace{1cm}
        {
            \fontsize{18}{0}
            \textbf{ՏԵՂԵԿԱՏՎԱԿԱՆ ՏԵԽՆՈԼՈԳԻԱՆԵՐԻ ԿՐԹԱԿԱՆ ԵՎ ՀԵՏԱԶՈՏԱԿԱՆ ԿԵՆՏՐՈՆ} \\
        }
        \vspace{1cm}
        {
            \fontsize{16}{0}
            \textbf{ՏԵՂԵԿԱՏՎԱԿԱՆ ՀԱՄԱԿԱՐԳԵՐԻ ԱՄԲԻՈՆ} \\
        }
        \vspace{1cm}
        {
            \fontsize{18}{0}
            \textbf{ՏԵՂԵԿԱՏՎԱԿԱՆ ՀԱՄԱԿԱՐԳԵՐԻ ԿԱՌԱՎԱՐՈՒՄ ԿՐԹԱԿԱՆ ԾՐԱԳԻՐ} \\
        }
        \vspace{2cm}
        {
            \fontsize{18}{0}
            \textbf{Մկրտչյան Ալբերտ Կարենի} \\
        }
        \vspace{1.5cm}
        {
            \fontsize{18}{0}
            \textbf{ՄԱԳԻՍՏՐՈՍԱԿԱՆ ԹԵԶ} \\
        }
        \vspace{1cm}
        {
            \fontsize{16}{0}
            \textbf{Ծրագրային կոդի հատկությունների հարցումների համակարգ՝ հիմնված ղեկավարման կախվածությունների գրաֆի վրա} \\
        }
        \vfill
        {
            \fontsize{14}{0}
            \textbf{\textit{«Տեղեկատվական համակարգերի կառավարում» մասնագիտությամբ ինֆորմատիկայի մագիստրոսի որակավորման աստիճանի հայցման համար}} \\
        }
        \vspace{1cm}
        {
            \fontsize{13}{0}
            \textbf{ԵՐԵՎԱՆ 2024} \\
        }
    \end{center}
\end{titlepage}

    {
	\large
	\raggedright
	\textbf{\textit{Ուսանող`}} \underline{\hspace{5cm}} \textbf{\textit{Մկրտչյան Ալբերտ}} \\
	\small
	\makebox[9cm][c]{\textit{Ստորագրություն}}
}

\vspace{2cm}
{
	\large
	\raggedright
	\textbf{\textit{Գիտական ղեկավար`}} \underline{\hspace{5cm}} \textbf{\textit{ֆ.մ.գ.թ., Ասլանյան Հայկ}} \\
	\small
	\makebox[8.1cm][r]{\textit{Ստորագրություն}}
}

\vspace{2cm}
{
	\large
	\raggedright
	\textbf{\textit{Գրախոս՝}} \underline{\hspace{5cm}} \textbf{\textit{ֆ.մ.գ.թ., դոցենտ, Մանուկյան Մանուկ}} \\
	\small
	\makebox[9cm][c]{\textit{Ստորագրություն}}
} \\


\vfill
{
	\large
	\raggedright
	\textbf{\textit{«Թույլատրել պաշտպանության»}}
} \\

\vspace{2cm}
{
	\large
	\raggedright
	\parbox[t]{3.9cm}{\textbf{\textit{Ամբիոնի վարիչ՝}}}
	\parbox[t]{2cm}{\underline{\hspace{4cm}} \small\makebox[4cm][c]{\textit{Ստորագրություն}}} \hfill
	\parbox[t]{8cm}{\raggedright\textbf{\textit{ՀՀ ԳԱԱ ակադեմիկոս, ֆ.մ.գ.դ., պրոֆեսոր, Սամվել Շուքուրյան}}} \\
}

\vspace{1.5cm}
{
	\raggedright
	\large
	\textit{«}\underline{\hspace{1.5cm}}\textit{»}\underline{\hspace{2.5cm}}20\underline{\hspace{0.5cm}}\textit{թ} \\
}

\newpage

    \clearpage
    {
	\thispagestyle{plain}
	\begin{center}
		\large
		\textbf{Համառոտագիր}
	\end{center}

	\begin{center}
		\vspace{0.5cm}

		\large
		\textbf{Ծրագրային կոդի հատկությունների հարցումների համակարգ հիմնված ղեկավարման կախվածությունների գրաֆի վրա}

		\vspace{0.2cm}
		\textbf{Система запроса свойств исходного кода на основе графа зависимостей управления}

		\vspace{0.2cm}
		\textbf{A system for querying source code properties based on a control dependency graph}

		\vspace{0.5cm}

	\end{center}
	Աշխատանքի շրջանակներում հետազոտվել են կախվածությունների գրաֆներ կառուցող ժամանակակից գործիքները։
	Գոյություն ունեցող գործիքները կախվածությունների գրաֆներ կառուցելու համար կատարում են ղեկավարման և
	տվյալների հոսքերի վերլուծություններ կոդի միջանկյալ ներկայացման հիման վրա։ Սակայն երբ ծրագրի նախնական
	կոդի մի մասը հասանելի չէ, մինջանկյալ ներկայացման թարգմանությունը հնարավոր չէ կատարել։ Աշխատանքում
	նախագծվել ու իրականացվել է գործիք, որը թույլ է տալիս շարահյուսորեն վերլուծվող նախնական կոդի համար
	կառուցել կախվածությունների գրաֆներ։
}

    \newpage
    \begin{center}
        \tableofcontents
    \end{center}

    \pagenumbering{arabic} % Միացնել էջերի համարակալումը
    \setcounter{page}{4} % համարակալումը սկսել 4-ից

    \clearpage
    \phantomsection
    \section*{\textbf{ՆԵՐԱԾՈՒԹՅՈՒՆ}}\label{sec:introduction}
    \addcontentsline{toc}{section}{ՆԵՐԱԾՈՒԹՅՈՒՆ}
    {
    Ծրագրային ապահովման ոլորտը շարունակաբար զարգանում և ընդլայնվում է, ինչի արդյունքում ծրագրային ապահովման ծավալներն
    անընդհատ աճում են: Անընդհատ աճող ծրագրային ապահովման ծավալների հետ մեկտեղ առաջանում է ծրագրային կոդի հատկությունների
    ավտոմատ վերլուծության գործիքների պահանջարկ: Ավտոմատացված վերլուծությունները թույլ են տալիս բացահայտել ծրագրի տարբեր
    կողմերը, ինչպիսիք են դասերի, ֆունկցիաների, փոփոխականների և հրահանգների կապերը, ինչպես նաև ստանալ տեղեկատվություն
    դրանց մասին: Նման վերլուծությունները կարևոր են ոչ միայն նոր ծրագրերի մշակման, այլև գոյություն ունեցող ծրագրերի
    պահպանման և զարգացման համար:
}


    \section{Արդիականություն}\label{sec:modernity}
        {

        \textbf{Relevance to Industry}

        The rapid growth of blockchain technology has underscored the need for advanced tools that can ensure the
        security and efficiency of smart contracts, particularly those written in Solidity. As blockchain applications
        increasingly handle financial transactions and sensitive data, the risk and impact of vulnerabilities have dramatically
        escalated. Our tool addresses this critical need by providing comprehensive analysis capabilities that enable developers
        to detect potential security flaws, optimize resource usage, and understand the intricate dependencies within
        their code. By leveraging a graph-based approach to analyze dependencies and properties of Solidity contracts,
        the tool offers a unique perspective that goes beyond the linear and often superficial checks performed by
        conventional linters and static analysis tools.

        \textbf{Applications in Smart Contract Development}

        \textbf{Security Audits}

        One of the primary applications of our tool is in conducting detailed security audits.
        The tool can identify common vulnerabilities such as reentrancy, unchecked external calls, and improper handling
        of exceptions.
        It does this by analyzing the flow of information across contracts and identifying tainted
        variables and potentially insecure dependencies.
        By providing these insights, the tool aids auditors in securing
        smart contracts before they are deployed on the blockchain, significantly reducing the risk of exploits and the
        associated financial losses.

        \textbf{Code Optimization and Gas Efficiency}

        Another vital application is in the optimization of smart contracts for better performance and lower transaction
        costs.
        Solidity developers are often challenged to write code that consumes less gas without compromising on
        functionality.
        Our tool assists in this by identifying inefficient patterns in code usage and suggesting more
        efficient alternatives.
        It also helps in refactoring code by revealing unnecessary dependencies and redundant
        data storage, which are often overlooked during manual reviews.

        \textbf{Educational and Training Purposes}

        Educators and trainers can utilize the tool to demonstrate best practices in smart contract development.
        By providing real-time feedback on code quality and security, the tool serves as an excellent resource for
        interactive learning and professional development in blockchain programming courses.
        It allows learners to see the immediate impact of their coding decisions and understand complex concepts
        like contract interdependencies and security practices in a hands-on manner.

        \textbf{ Integration with Development Pipelines}

        For development teams, integrating this tool into their CI/CD pipelines ensures that every piece of code is
        automatically analyzed for potential issues before deployment. This integration helps maintain high standards
        of code quality and security throughout the development process, greatly reducing the likelihood
        of introducing vulnerabilities into the live environment.

        \textbf{Contribution to the Blockchain Ecosystem}

        By providing a robust tool for analyzing and querying properties of Solidity code based on program
        dependencies graphs, we contribute to the broader blockchain ecosystem by enhancing the security
        and reliability of smart contract deployments. This contribution is crucial not only in fostering trust
        in blockchain-based applications but also in advancing the technology towards wider adoption in
        various sectors including finance, healthcare, and government.

        In summary, our tool is not just a technical solution but a critical asset in the blockchain development lifecycle, addressing the pressing needs for security, efficiency, and quality assurance in smart contract development. Its applications extend from practical, day-to-day development tasks to strategic audits and educational initiatives, making it an indispensable resource for developers, auditors, and educators alike.
    }



    \section{Խնդրի դրվածքը}\label{sec:problemFormulation}
    {
	Նախագծել և իրականացնել ծրագրային կոդի հատկությունների հարցումների համակարգ,
	որը կվերլուծի և կտրամադրի տեղեկատվություն դասերի, ֆունկցիաների,
	հրահանգների, փոփոխականների և դրանց կապերի մասին:
}


    \section{Գոյություն ունեցող գործիքներ}\label{sec:existingTools}
        {
        Աշխատանքի սկզբնական փուլում կատարվել է արդեն գոյություն ունեցող կոդի ստատիկ վերլուծություն կատարող գործիքների
        ուսումնասիրություն։ Արդյունքում հետազոտվել են առաջատար գործիքների մշակված ալգորիթմների թերություններն ու
        առավելությունները, որոշ գործիքներ փորձարկվել են Ջուլիետ թեստային հավաքածույում՝ CWE-401 տիպի սխալներ\cite{CWE401} գտնելու համար։

        \subsubsection{SMOKE}
        SMOKE\cite{Fan2019}-ի ալգորիթմը կազմված է երկու հիմնական փուլերից՝ բարձր ճշտության և մասշտաբայնության հասնելու համար։
        Առաջին փուլում այն օգտագործում է պարզ, բայց ոչ ճշգրիտ վերլուծություն՝ հիշողության արտահոսքի բոլոր հնարավոր ուղիները
        հայտնաբերելու համար և զտում է այն ուղիները, որոնք չեն կարող հանգեցնել արտահոսքի: Այս նպատակով նախ օգտագործվում է նոսր
        արժեքների կախվածության գրաֆը, ապա կառուցվում է օգտագործման կախվածության գրաֆը(UFG): UFG-ն պարունակում է վերլուծության
        համար բավարար ինֆորմացիա բոլոր դինամիկ հիշողության օբյեկտների մասին։ UFG-ի յուրաքանչյուր կող համապատասխանեցվում է պայմանների հետ,
        որոնցից կախված ուղղորդվում է ծրագրի աշխատանքի ընթացքը։
        Նկար \ref{fig:figure1}-ում(վերցված է\cite{Fan2019}-ից) պատկերված է UFG-ի օրինակ։ UFG-ն բացահայտ նկարագրում է ցուցիչի արժեքների հնարավոր
        ընթացքն ստեղծելով սահմաններից դուրս գտնվող գագաթ(p@s8)։ Դա նշանակում է, որ դինամիկ հիշողության օբյեկտը, որն
        օգտագործվում էր վերը նշված ցուցիչի միջոցով այլևս հասցեավորված չէ։

        \begin{figure}[h]
            \centering
            \includegraphics[width=0.6\textwidth]{pic1}
            \caption{Դինամիկ հիշողության արտահոսքի օրինակ}
            \label{fig:figure1}
        \end{figure}

        Երկրորդ փուլում այն օգտագործում է արդեն ստացված UFG֊ի առավել ճշգրիտ վերլուծություն։ Սկզբում որոնում է բոլոր
        ճանապարհները, որոնք չեն պարունակում դինամիկ հիշողություն օգտագործող օպերացիաներ։ Այնուհետև հայտնաբերված յուրաքանչյուր
        ճանապարհի համար կիրառում է Z3 գործիքը\cite{Z3}՝ նրանց իրագործելիությունը ստուգելու համար: Այս գործընթացն անհրաժեշտ է
        ճանապարհների զգայունությունն ապահովելու և կեղծ հայտնաբերված արտահոսքերը զտելու համար։ Գործիքն ունի որոշ սահմանափակումներ՝
        \begin{enumerate}[itemsep=1mm]
            \item Վերլուծությունն արվում է դաշտերի հանդեպ ոչ զգայուն։
            \item Ցուցիչների վերլուծությունը ճշգրիտ չէ։
            \item Որոշ ճանապարհներ անիրագործելի են բարդ թվաբանական և միջֆունկցիոնալ տվյալների կախվածությունների պատճառով։
            \item Հաշվի չի առնվում թվաբանական գործողությունները ցուցիչների հետ, free(p + y) արտահայտությունը համարժեք է համարվում free(p)֊ին, ինչն ակնհայտ սխալ է։
            \item Հաշվի են առվում միայն այն դեպքերը, երբ դինամիկ հիշողության առանձնացման գործողությունը բարեհաջող է անցնում։
        \end{enumerate}

        SMOKE֊ը ծրագրավորված է LLVM-ի վերին մակարդակում և միայն բինար տարբերակն է հասանելի\cite{SMOKE}:

        \subsubsection{PCA}
        CA\cite{Li2020}֊ն առաջին հերթին թարգմանում է նախնական կոդը LLVM IR֊ի։ Այնուհետև, օգտագործելով LLVM֊ի gold plugin֊ը
        բոլոր IR ֆայլերը հավաքում է մեկում։ Ապա այն օգտագործում է Անդերսենի ցուցիչների վերլուծության գործիքը\cite{Andersen}
        և հավաքում ցուցիչների մասին ինֆորմացիա, մասնավորապես, թե որ դինամիկ հիշողության օբյեկտի վրա է հղված այս կամ այն ցուցիչը։
        Այդ ինֆորմացիայի վրա հիմնվելով, կառուցվում են ֆունկցիաների կանչերի կախվածության և ղեկավարման կախվածության գրաֆները։
        Եւ վերջում, արդեն ունենալով համապատասխան գրաֆներն ու ինֆորմացիան, կառուցվում է միջֆունկցիոնալ տվյալների կախվածության գրաֆ(DDG)։

        Դինամիկ հիշողության արտահոսքի հայտնաբերման նպատակով PCA֊ը յուրաքանչյուր դինամիկ հիշողություն առանձնացնող ինստրուկցիայի(A)
        համար հավաքում է բոլոր նրանից հասանելի գագաթները(N) DDG֊ում: Եթե N֊ը առանձնացված հիշողությունն ազատող ինստրուկցիա
        չի ներառում, ապա պնդում է, որ տեղի ունի հիշողության արտահոսք։

        Որոշ դինամիկ հիշողության օբյեկտների կարող են հետևել մեկից ավելի դրանք ազատող ինստրուկցիաներ DDG֊ում։
        Նման դեպքերում, A֊ն կհամարվի ազատված, եթե գոյություն ունի ղեկավարման կախվածության ճանապարհ A֊ից դեպի այն ազատող
        ինստրուկցիան։ Հակառակ դեպքում նույնպես պնդում է, որ տեղի ունի հիշողության արտահոսք։ Գործիքն ունի հետևյալ սահմանափակումները՝
        \begin{enumerate}[itemsep=1mm]
            \item Վերլուծությունն արվում է հոսքի, դաշտի և համատեքստի հանդեպ ոչ զգայուն։
            \item Այն օգտագործում է LLVM gold plugin֊ը IR ֆալերը մեկում հավաքելու համար, այնուհետև կառուցում է DDG, ինչը գործիքը դարձնում է ոչ մասշտաբային։
            Գործիքի նախնական կոդը հասանելի է\cite{PCA}։
        \end{enumerate}

        \subsubsection{SVF}
        SVF\cite{Sui2016}֊ն հիմնված է LLVM֊ի վերին մակարդակում։ Առաջին քայլում այն թարգմանում է նախնական կոդը LLVM IR֊ի։
        Այնուհետև, օգտագործելով LLVM֊ի gold plugin֊ը բոլոր IR ֆայլերը հավաքում է մեկում։ SVF֊ն օգտագործում է որոշ ցուցիչների
        անալիզատորներ, ցուցիչների համար համապատասխան ինֆորմացիա հավաքելու նպատակով, մասնավորապես, թե որ դինամիկ հիշողության
        օբյեկտի վրա է հղված այս կամ այն ցուցիչը։ Անալիզատորներցի մեկն Անդերսենի ցուցիչների անալիզատորն է\cite{Andersen}։
        Օգտագործելով LLVM IR֊ը և ցուցիչների մասին հավաքված ինֆորմացիան՝ գործիքը կառուցում է ղեկավարման կախվածության գրաֆ(CFG)
        և հավաքում է հիշողության ստատիկ առանձին վերագրման մասին ինֆորմացիան(SSA - Static Single Assignment)։
        Յուրաքանչյուր VFG֊ի գագաթ իրենի ներկայացնում է ծրագրի որևէ ինստրուկցիա, իսկ գագաթների միջև կողերը տեղադրվում են
        օգտվելով օգտագործման֊հայտարարման և ցուցիչների մասին հավաքված ինֆորմացիայից։

        Բացի այդ, SVF-ն ապահովում է հիշողության տարածքների տարանջատում, ինչը թույլ է տալիս օգտվողներին հիշողությունը բաժանել
        հավաքածուների: Սա օգտակար է մեծածավալ ծրագրերը վերլուծելու համար, եթե հաշվի է առնվում հիշողության որոշակի տարածք:

        Տարբեր ստուգման գործիքներ կարող են իրականացվել VFG֊ի հիման վրա, որը հասանելի է օգտվող ծրագրերի համար:
        Հիշողության արտահոսքի հայտնաբերումը համարվում է աղբյուր-ստացողի խնդիր (յուրաքանչյուր հիշողության հատկացում
        յուրաքանչյուր ուղու վրա պետք է հասնի իր ազատմանը): Ստորև նշված են գործիքի որոշ սահմանափակումներ՝
        \begin{enumerate}[itemsep=1mm]
            \item Այն օգտագործում է LLVM gold plugin֊ը IR ֆալերը մեկում հավաքելու համար, այնուհետև կառուցում է DDG,
            ինչը գործիքը դարձնում է ոչ մասշտաբային։
            \item Վերլուծությունն արվում է ճանապարհների և դաշտերի հանդեպ ոչ զգայուն։
        \end{enumerate}
    }


    \section{Ծրագրային կոդի հատկությունների հարցումների համակարգի նախագծում}\label{sec:queryEngineDesign}
    {
	Մի քանի խոսքով ընդհանուր պատկերի մասին ակնարկ։ Նկար \ldots
	Օգտագործողին հարկավոր չէ իմանալ ստատիկ վերլուծություն, բավարար է միայն տվյալ ծրագրային լեզվի իմացությունը։

	\subsection{Տվյալների հավաքագրում}\label{subsec:dataCollection}
	{
    \subsection{Տվյալների հավաքագրում}\label{subsec:dataCollection}
    Ծրագրային կոդի հատկությունների հարցումների համակարգի համար էական դեր ունի տվյալների հավաքագրման փուլը
    (Նկար \ref{fig:figure5}): Այս փուլում  կարևոր է տվյալներ հավաքագրող համակարգի
    օգտագործումը: Տվյալներ հավաքագրող համակարգի նպատակն է վերլուծել ծրագրային կոդը և հավաքագրել անհրաժեշտ
    տեղեկություններ՝ ներառյալ ծրագրի կառուցվածքի, ղեկավարման հոսքի և ներքին փոխկապակցվածությունների վերաբերյալ։

    \begin{figure}[h]
        \centering
        \includegraphics[width=1\textwidth]{pic5}
        \caption{Փուլ 1, տվյալների հավաքագրում}
        \label{fig:figure5}
    \end{figure}

    Տվյալներ հավաքագրող համակարգը մուտքում ստանում է աղբյուր կոդի ֆայլերը և իրականացնում վերլուծություններ՝
    ընդգրկելով հետևյալ տեղեկությունների հավաքագրումը.
    \begin{enumerate}
        \item Դասերի մասին տեղեկություններ՝ ներառյալ յուրաքանչյուր դասի անունը, դասի դաշտերը,
        աղբյուր կոդի տողերի համարները, ժառանգականության կապերը, ֆունկցիաները և այլն,
        \item Ֆունկցիաների մասին տեղեկություններ՝ ներառյալ յուրաքանչյուր ֆունկցիայի անունը, վերասահմանված ֆունկցիա լինելը,
        աղբյուր կոդի տողերի համարները, հայտարարող դասի անունը, կանչվող ֆունկցիաները, արգումենտները, լոկալ փոփոխականները,
        օգտագործվող և սահմանվող դաշտերը, տեսանելիությունը և այլն,
        \item Հրահանգների մասին տեղեկություններ՝ ներառյալ յուրաքանչյուր հրահանգների տիպը, աղբյուր կոդի տողերը,
        կանչվող ֆունկցիաները, օպերանդները, օգտագործվող և սահմանվող փոփոխականները և այլն:
    \end{enumerate}

    Այս տվյալները պահվում են կառուցված տվյալների բազայում՝ հետագա վերլուծությունների և հարցումների համար
    օգտագործելու նպատակով: Դրանք հիմք են հանդիսանում ծրագրային կոդի հատկությունների հարցումների համակարգի համար:

    {
    \subsubsection{Տվյալների բազայի նախագծում}\label{subsubsec:database}

    Հարցումները արդյունավետ կատարելու նպատակով նախագծվել է տվյալների բազա՝ հաշվի առնելով ռելացիոն ու ոչ ռելացիոն
    բազաների հատկությունները։ Տվյալների բազան ներառում է աղբյուր կոդի հիմնական տարրերի՝ դասերի, դասերի անդամ դաշտերի
    և հրահանգների մասին տեղեկություններ (Նկար \ref{fig:figure7}):

    \begin{figure}[h]
        \centering
        \includegraphics[width=0.8\textwidth]{pic7}
        \caption{Տվյալների բազայի կառուցվածքը}
        \label{fig:figure7}
    \end{figure}
}

%    {
    \subsubsection{Բաց կոդով հասանելի նախագծերի հավաքագրում}\label{subsubsec:sourceCodes}
}
}

	\subsection{Հարցումների համակարգ}\label{subsec:queries}
	{
    \subsection{Հարցումների համակարգ}\label{subsec:queries}

    Երկրորդ փուլում նախագծվել է բուն հարցումների համակարգը:
    Նախագծված համակարգը հանդիսանում է ծրագրային կոդի հատկությունների վերլուծության կարևոր բաղկացուցիչ մասը:
    Հարցումների համակարգը ճկուն է, ընդլայնելի և հեշտ օգտագործվող:
    Այս համակարգը թույլ է տալիս ծրագրավորողներին, նույնիսկ առանց խորը վերլուծական գիտելիքների,
    արագ և ճշգրիտ ստանալ տեղեկատվություն իրենց ծրագրերի կառուցվածքի և հատկությունների մասին:

    Նախագծված API-ի միջոցով օգտագործողները կարող են կատարել հարցումներ, որի արդյունքում համակարգը տվյալների բազայից վերցնում է անհրաժեշտ
    տվյալներ, կատարում համապատասխան վերլուծություններ և տալիս է հարցմումների պատասխանները (Նկար \ref{fig:figure6}):

    \begin{figure}[h]
        \centering
        \includegraphics[width=1\textwidth]{pic6}
        \caption{Փուլ 2, Հարցումների համակարգ}
        \label{fig:figure6}
    \end{figure}

    {
    Գրել բաժանումների մասին
    \subsubsection*{Կլասներ}
    Կլասների հարցումների մասին խոսել։

    \subsubsection*{Ֆունկցիաներ}
    Ֆունկցիաների հարցումների մասին խոսել։

    \subsubsection*{Ինստրուկցիաներ}
    Ինստրուկցիաների հարցումների մասին խոսել։

    \subsubsection*{Վիճակի փոփոխության վերլուծություն}
    Ավելացնել ալգորիթմերի բարդությունը։
}
}
}



    \section{Ծրագրային կոդի հատկությունների հարցումների համակարգի իրականացում}\label{sec:queryEngineImplementation}
    {
	Գործիքների ճշտությունը գնահատելու համար մշակվել է ավտոմատ թեստային համակարգ:
}



    \section{Դինամիկ հիշողության արտահոսքի սխալների հայտնաբերում}\label{sec:bugDetection}
    \clearpage
\section{Դինամիկ հիշողության արտահոսքի սխալների հայտնաբերում}\label{sec:bugDetection}
Ծրագրային կոդի հատկությունների հարցումների համակարգը օգտագործվել է դինամիկ հիշողության
արտահոսքեր (memory leaks)\cite{MEMORYLEAK} հայտնաբերելու համար: Այդ նպատակով, համակարգի API-ի օգտագործմամբ
իրականացվել է համապատասխան ստուգիչ (checker) (Տես Նկար \ref{fig:figure3}): Ստուգիչի աշխատանքը հիմնված է հետևյալ ալգորիթմի վրա.
\begin{enumerate}
    \item Գտնել malloc, calloc ֆունկցիաների կանչերի հրահանգները,
    \item Յուրաքանչյուր malloc, calloc կանչի համար որոշել նպատակային օպերանդը (destination),
    \item Յուրաքանչյուր նպատակային օպերանդից գտնել կախվածություն ունեցող և free կանչող հրահանգները (forward\_df\_free\_instructions),
    \item Յուրաքանչյուր ճանապարհի համար, որ սկսվում է malloc, calloc կանչից և ավարտվում է ծրագրի վերջում, ստուգել,
    արդյոք այդ ճանապարհը պարունակում է forward\_df\_free\_instructions հրահանգներից որևէ մեկը:
    \item Եթե ոչ, արձանագրել դինամիկ հիշողության արտահոսք, եթե ճանապարհն իրական է:
\end{enumerate}

\begin{figure}[h]
    \centering
    \includegraphics[width=1\textwidth]{pic3}
    \caption{Հարցումների համակարգի API-ի օգտագործմամբ գրված դինամիկ հիշողության արտահոսքի ստուգիչ}
    \label{fig:figure3}
\end{figure}

Հարցումների համակարգի օգտագործմամբ ստեղծված ստուգիչը թույլ է տալիս ավտոմատ կերպով հայտնաբերել
այնպիսի դեպքեր, երբ ծրագրի աշխատանքի ընթացքում հիշողությունը չի ազատվում ամբողջությամբ: Սա կարևոր խնդիր է ծրագրային
ապահովման որակի և արդյունավետության տեսանկյունից, քանի որ չազատված հիշողությունը կարող է հանգեցնել ծրագրի
կայունության և կատարողականության նվազեցմանը:
{
    \subsection{Թեստավորում}\label{subsec:testing}
    Մշակված գործիքը թեստավորվել և համեմատվել է աշխարհում արդեն գոյություն ունեցող այլ գործիքների հետ։ Ինչպես նաև գործիքի
    միջոցով թեստավորվել են բաց կոդով հասանելի ավելի քան 100 պրոեկտ, որոնք իրականացված են մեծամասամբ C ծրագրավորման լեզվով։

    \subsubsection{Արդյունքների համեմատումը Juliet թեստերի հավաքածույի վրա}
    MLH֊ը թեստավորվել է Juliet թեստերի հավաքածույի վրա, որը նախատեսված է ծրագրային ապահովման գործիքների թեստավորման և
    արդյունքների գնահատման համար։ Այն Software Assurance Metrics and Tool Evaluation(SAMATE)\cite{SAMATE} նախագծի մասն է
    կազմում, որը մշակվել է National Institue of Standards and Technology(NIST)֊ի կողմից։ Ջուլիետ թեստերի հավաքածուն
    ներառում է ավելի քան 120,000 թեստերի օրինակներ առանձնացված զանազան խնդիրների համար ինչպիսիք են դինամիկ հիշողության
    արտահոսքը, բուֆերի գերհագեցումը և այլն։ Այն ներառում է թեստեր C/C++, Java և Ada լեզուներով:
    Թեստավորման համար հավաքածուից առանձնացվել են միայն դինամիկ հիշողության արտահոսքի համար նախատեսված
    օրինակները (CWE401\_Memory\_Leak\cite{CWE401}): Մշակվել է թեստավորման համակարգ, որը գնահատում է ունեցած գործիքների
    արդյունավետությունը վերը նշված հավաքածուի համար։ Ստորև ներկայացված է առաջատար այլ գործիքների և
    նոր մշակված գործիքի արդյունավետության աղյուսակը (Աղյուսակ\ref{tab:juliet_test}):
    \begin{table}[h!]
    \centering
    \begin{tabularx}{\textwidth}{|*{6}{>{\centering\arraybackslash}X|}}
        \hline
        \textbf{Name} & \textbf{True Positives} & \textbf{True Negatives} & \textbf{False Positives} & \textbf{False Negatives} & \textbf{F1 score} \\
        \hline
        \textbf{CSA} & 536 & 4481 & 125 & 332 & 0.7011 \\
        \hline
        \textbf{Infer} & 262 & 4392 & 214 & 606 & 0.3899 \\
        \hline
        \textbf{SMOKE} & 496 & 4510 & 96 & 372 & 0.6795 \\
        \hline
        \textbf{PCA} & 486 & 4342 & 264 & 382 & 0.6007 \\
        \hline
        \textbf{SVF} & 452 & 4168 & 438 & 416 & 0.5142 \\
        \hline
        \rowcolor{yellow!100} \textbf{MLH} & 868 & 4606 & 0 & 0 & 1 \\
        \hline
    \end{tabularx}
    \caption{Juliet թեստերի հավաքածույի վրա համեմատման արդյունքների}
    \label{tab:juliet_test}
\end{table}
}

{
    \subsection{Արդյունքներ}\label{subsec:results}
}

    \section{Հետագա աշխատանք}\label{sec:furtherWork}
    {
	Նախատեսվում է ... \\
	\\
	Այժմ աշխատանք է տարվում ...
}

    \clearpage
    \phantomsection
    \section*{Եզրակացություն}\label{sec:conclusion}
    \addcontentsline{toc}{section}{Եզրակացություն}
    {
	\begin{enumerate}
		\item
		Նախագծվել և իրականացվել է ծրագրային կոդի հատկությունների հարցումների համակարգ,

		\item
		Օգտագործելով հարցումների համակարգը` ստեղծվել է դինամիկ հիշողության արտահոսքի սխալների հայտնաբերման ալգորիթմ

        {
			\begin{enumerate}
				\item
				Թեստավորվել են C լեզվով գրված` բաց կոդով հասանելի 100 նախագծերի վրա (top 100 by github stars), որի արդյունքում հայտնաբերվել են սխալներ։
			\end{enumerate}
		}

	\end{enumerate}
}


    \clearpage
    \phantomsection
    \addcontentsline{toc}{section}{Գրականություն}
    \bibliographystyle{plainnat}
    \bibliography{bibliography}
\end{document}